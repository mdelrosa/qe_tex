% 02_previous_work_first.tex

\blindtext 
\blindtext

\subsection{Related Work}

\blindtext
\blindtext
\blindtext

\subsection{Methods}

\blindtext

\subsubsection{Notation}

For an example of a table, see Table~\ref{table:notation}.

\begin{table}[htb]
	%	\small
	\caption{Notations}
	\label{table:notation}
	\centering
	\begin{tabular}[]{c|l}
		\toprule
		Variable & Definition \\
		\midrule
		$M$ & Number of input hydrological variables denoted in Fig.~\ref{fig:ann_inputs}\\
		\hline
		$N$ & Number of data samples, or days, in dataset \\
		\hline
		$T$ & Number of days' data used for estimation \\
		\hline
		$T_r$ & Dimension of data after pre-processing\\
		\hline
		$z_{n}$ & Time series used for estimating salinity level on day $n$, size is ${\rm I\!R}^{M\times T}$ \\
		\hline
		$x_{n}$ & Pre-processed time series with size ${\rm I\!R}^{M\times T_r}$ for day $n$\\
		\hline
		$f$ & A convolutional filter with size ${\rm I\!R}^{M \times T \times T_r}$ \\
		\hline
		$y_{n}$ & ANN-estimated salinity level for one or more locations on day $n$ \\
		\bottomrule
	\end{tabular}
\end{table}

% \begin{figure}[htb]
% 	\centering
% 	\includegraphics[width=.7\textwidth]{math_pipeline.png}
% 	\medskip
% 	\caption{Pipeline for ANNs in \cite{jayasundara2020artificial} with mathematical notations}
% 	\label{fig:math_pipeline}
% \end{figure}

\subsubsection{Method Subsection \#1}
\label{sect:methods_sub1}

\blindtext
\blindtext

\subsubsection{Methods Subsection \#2}
\label{sect:methods_sub2e

\blindtext

\begin{equation}
x_{n,i}^{(m)} = \sum_{j=1}^{T}z_{n-j+1}^{(m)}\times f_{j,i}^{(m)},
\label{eq:conv}
\end{equation}

\blindtext
